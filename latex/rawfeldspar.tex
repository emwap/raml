\documentclass[natbib]{sigplanconf}

\usepackage{amsmath}
\usepackage{amsthm}
\usepackage{amssymb}
\usepackage{stmaryrd}
\usepackage{haskell}
\usepackage{url}
\usepackage{natbib}
\usepackage{proof}

\newcommand{\kw}[1]{\mbox{\fontfamily{cmr}\fontseries{b}\selectfont#1}}

\newtheorem{theorem}{Theorem}[section]
\newtheorem{definition}[theorem]{Definition}
\newtheorem{example}[theorem]{Example}
\newtheorem{prop}[theorem]{Proposition}
\newtheorem{lemma}[theorem]{Lemma}
\newtheorem{corollary}[theorem]{Corollary}

\newcommand{\hinfer}[3]{\vcenter{\infer[\DT{#1}]{#2}{#3}}}
\newcommand{\ENV}{\Gamma}
\newcommand{\beval}{\mapsto} % FIXME: Wrong arrow.
\newcommand{\ent}{\vdash}
\newcommand{\cost}{\alt}
\newcommand{\spacer}{\,\,\,\,} % FIXME: Wrong spacer

\newcommand{\DT}[1]{\mbox{\sc #1}}
\newcommand{\alt}{\;\:\vert\;\:}

\newcommand{\KPAR}{K^{par}}
\newcommand{\KFOR}{K^{for}}
\newcommand{\K}[1]{K_{#1}}

\newcommand{\IF}{\kw{if}\,}
\newcommand{\THEN}{\,\kw{then}\,}
\newcommand{\ELSE}{\,\kw{else}\,}
\newcommand{\LET}{\kw{let}\,}
\newcommand{\IN}{\,\kw{in}\,}
\newcommand{\PARALLEL}{\kw{parallel}\,}
\newcommand{\FOR}{\kw{for}\,}
\newcommand{\INDEX}{\,\kw{!}\,\,}

\newcommand{\TRUE}{True}
\newcommand{\FALSE}{False}
\newcommand{\ARRAY}[1]{[#1]}

\newcommand{\FST}{\kw{fst}\,}
\newcommand{\SND}{\kw{snd}\,}

\begin{document}

%\special{papersize=8.5in,11in}
%\setlength{\pdfpageheight}{\paperheight}
%\setlength{\pdfpagewidth}{\paperwidth}


\conferenceinfo{XXX}{YYY}
\CopyrightYear{2012}
\copyrightdata{ZZZ}

\titlebanner{Draft: \today}        % These are ignored unless
\preprintfooter{short description of paper}   % 'preprint' option specified.

\title{Raw Feldspar}

\authorinfo{Everybody \and us}
           {Chalmers, SICS and Ericsson}
           {\{name1, name2\}@chalmers.se, etc}

\maketitle

\begin{abstract}
Raw Feldspar.
\end{abstract}

\section{Introduction}

\section{Language}
\label{sec:lang}

Feldspar core is a first-order strict functional language 
with let-bindings and primitive arrays. Its syntax for expressions,
values and patterns is shown in Figure \ref{fig:language}.

\begin{figure}
\begin{tabular}{lll}
\multicolumn{3}{l}{Expressions} \\ \hline \\
  $e$ &::= & $n \alt x \alt \TRUE \alt \FALSE  \alt x_1
      \oplus  x_2
      \alt  f(\overline{x})$ \\
 &  $\alt$ & $(x_1, x_2) \alt \FST x \alt \SND x$ \\ 
 &  $\alt$ & $\LET x = e_1 \IN e_2 \alt \IF x \THEN e_t \ELSE e_f$ \\ 
 &  $\alt$ & $\PARALLEL x\, i.e \alt \ARRAY{\overline{v}} \alt  x_1 \INDEX x_2 \alt
              \FOR x\, y\, i.s.e$\\
\\
\multicolumn{3}{l}{Values} \\ \hline \\
  $v$ &::= & $n  \alt \TRUE \alt \FALSE \alt \ARRAY{\overline{v}}$ \\
\end{tabular}  
\caption{The language}
\label{fig:language}
\end{figure}

The language contains integer values $n$ and arithmetic
operations $\oplus$, although these meta-variables can preferably be
understood as ranging over primitive values in general and arbitrary
operations on these. We let $+$ denote the semantic meaning of
$\oplus$.

A program is an expression with no free variables and all function
names defined in XXX.  The intended operational
semantics is given in Figure \ref{fig:redsem}, where
$[\overline{e}/\overline{x}]e'$ is the capture-free substitution of
expressions $\overline{e}$ for variables $\overline{x}$ in $e'$.

\begin{figure}
$\hinfer{Par}{\ENV \ent \PARALLEL x\,i.e \beval \ARRAY{v_0\ldots v_{n-1}} \cost \KPAR}{\ENV(x) = n \spacer \ENV {i \mapsto j} \ent e \beval v_j \cost \K{j} \spacer j=0\ldots n-1}$

$\hinfer{For}{\ENV \ent \FOR x\,y\, i.s.e \beval v_n \cost \KFOR}{\ENV(x) = n \spacer \ENV(y) = v_0 \spacer \ENV {i \mapsto j, s \mapsto v_j} \ent e \beval v_{j+1} \cost \K{j} \spacer j=0\ldots n-1}$
\caption{Reduction semantics}
\label{fig:redsem}
\end{figure}

\section{Related Work}

\section{Conclusions}

\acks


\bibliographystyle{plainnat}

\bibliography{rawfeldspar}
\end{document}

