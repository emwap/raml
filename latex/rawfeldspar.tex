\documentclass[natbib]{sigplanconf}

% Issues/TODOs
%
% [Weird] Arrays can only be constructed through the parallel primitive, but 
% they need to be present in grammar and values for the lanugage so that 
% parallel evaluates to something. Strange asymmetry.
%
% [Notation] We should not overload Gamma for the environments for the 
% operational semantics and typing rules.
%
% [Notation] Find a neat type symbol instead of ``Arr t'' for arrays.
%
% [Notation] Find a neat symbol instead of [vs] for arrays.
%
% [Notation] Find a syntactically small way to extend environments
%
% [Notation] Does the Index type need to be called Index, or should we call
% it Ix/I_x instead?


% Misc comments/insights
%
% One reason that Hoffmann et al. separate Q syntactically in the type 
% environment is that the rules would look like crap from all the pairs
% otherwise
%
% It's tempting to use e_f for the false branch in if in the language
% grammar, but that clashes with e_f that we use to denote the body of 
% the function f.


\usepackage{amsmath}
\usepackage{amsthm}
\usepackage{amssymb}
\usepackage{stmaryrd}
\usepackage{haskell}
\usepackage{url}
\usepackage{natbib}
\usepackage{proof}

\newcommand{\kw}[1]{\mbox{\fontfamily{cmr}\fontseries{b}\selectfont#1}}

\newtheorem{theorem}{Theorem}[section]
\newtheorem{definition}[theorem]{Definition}
\newtheorem{example}[theorem]{Example}
\newtheorem{prop}[theorem]{Proposition}
\newtheorem{lemma}[theorem]{Lemma}
\newtheorem{corollary}[theorem]{Corollary}

\newcommand{\hinfer}[3]{\vcenter{\infer[\DT{#1}]{#2}{#3}}}
\newcommand{\ENV}{\Gamma}
\newcommand{\beval}{\Downarrow} % FIXME: Wrong arrow.
\newcommand{\ent}{\vdash}
\newcommand{\cost}{\alt}
\newcommand{\nl}{\vspace{2.5ex} \\}
\newcommand{\il}{\vspace{0.4ex} \\}
\newcommand{\spc}{\hspace{5ex}}

\newcommand{\shift}{\triangleleft}

\DeclareMathOperator{\dom}{\it dom}

\newcommand{\DT}[1]{\mbox{\sc #1}}
\newcommand{\alt}{\;\:\vert\;\:}

% Costs for operations.
\newcommand{\KBOOL}{K^{bool}}
\newcommand{\KINT}{K^{int}}
\newcommand{\KIX}{K^{ix}}
\newcommand{\KARR}{K^{arr}}
\newcommand{\KVAR}{K^{var}}
\newcommand{\KOP}{K^{binop}}
\newcommand{\KPAIR}{K^{pair}}
\newcommand{\KFST}{K^{fst}}
\newcommand{\KSND}{K^{snd}}
\newcommand{\KCONT}[1]{K^{ifT}_{#1}}
\newcommand{\KCONF}[1]{K^{ifF}_{#1}}
\newcommand{\KLET}[1]{K^{let}_{#1}}
\newcommand{\KPAR}[1]{K^{par}_{#1}}
\newcommand{\KFOR}[1]{K^{for}_{#1}}
\newcommand{\KAPP}[1]{K^{app}_{#1}}
\newcommand{\KINDEX}{K^{index}}
\newcommand{\K}[1]{K_{#1}}

% Cost addition
\newcommand{\KADD}{+} % FIXME: Wrong symbol.

% Keywords.
\newcommand{\IF}{\kw{if}\,}
\newcommand{\THEN}{\,\kw{then}\,}
\newcommand{\ELSE}{\,\kw{else}\,}
\newcommand{\LET}{\kw{let}\,}
\newcommand{\IN}{\,\kw{in}\,}
\newcommand{\PARALLEL}{\kw{parallel}\,}
\newcommand{\FOR}{\kw{for}\,}
\newcommand{\INDEX}{\,\kw{!}\,\,}
\newcommand{\FST}{\kw{fst}\,}
\newcommand{\SND}{\kw{snd}\,}

% Values
\newcommand{\TRUE}{True}
\newcommand{\FALSE}{False}
\newcommand{\ARRAY}[1]{[#1]}
\newcommand{\PAIR}[2]{(#1, #2)}


% Type names
\newcommand{\TINDEX}{Index}
\newcommand{\TINT}{Int}
\newcommand{\TBOOL}{Bool}
\newcommand{\TPAIR}[2]{#1 \times #2}
\newcommand{\TARRAY}{Arr\,}
\newcommand{\TFUN}{\rightarrow}


\newcommand{\TENV}{\Gamma} % FIXME: Clashes with \ENV

\begin{document}

%\special{papersize=8.5in,11in}
%\setlength{\pdfpageheight}{\paperheight}
%\setlength{\pdfpagewidth}{\paperwidth}


\conferenceinfo{XXX}{YYY}
\CopyrightYear{2012}
\copyrightdata{ZZZ}

\titlebanner{Draft: \today}        % These are ignored unless
\preprintfooter{short description of paper}   % 'preprint' option specified.

\title{Raw Feldspar}

\authorinfo{Everybody \and us}
           {Chalmers, SICS and Ericsson}
           {\{name1, name2\}@chalmers.se, etc}

\maketitle

\begin{abstract}
See the top of the tex-file for some more outstanding issues.
\end{abstract}

\section{Introduction}

\section{Language}
\label{sec:lang}

Feldspar core is a first-order strict functional language 
with let-bindings and primitive arrays. Its syntax for expressions,
values and patterns is shown in Figure \ref{fig:language}.

\begin{figure}
\begin{tabular}{lll}
\multicolumn{3}{l}{Expressions} \\ \hline \\
  $e$ &::= & $n \alt x \alt \TRUE \alt \FALSE  \alt x_1
      \oplus  x_2
      \alt  f(\overline{x})$ \\
 &  $\alt$ & $(x_1, x_2) \alt \FST x \alt \SND x$ \\ 
 &  $\alt$ & $\LET x = e_1 \IN e_2 \alt \IF x \THEN e_1 \ELSE e_2$ \\ 
 &  $\alt$ & $\ARRAY{\overline{v}} \alt \PARALLEL x\, i.e \alt x_1 \INDEX x_2 \alt
              \FOR x\, y\, i.s.e$\\
\\
\multicolumn{3}{l}{Values} \\ \hline \\
  $v$ &::= & $n  \alt i \alt \TRUE \alt \FALSE \alt \PAIR{v}{v'} \alt \ARRAY{\overline{v}}$ \\
\\
\multicolumn{3}{l}{Types} \\ \hline \\
  $t$ &::= & $\TINT \alt \TINDEX \alt \TBOOL \alt \TPAIR{t}{t'}  \alt \TARRAY t \alt \overline{t} \TFUN t$ \\
\end{tabular}  
\caption{The language}
\label{fig:language}
\end{figure}

The language contains integer values $n$ and arithmetic
operations $\oplus$, although these meta-variables can preferably be
understood as ranging over primitive values in general and arbitrary
operations on these. We let $+$ denote the semantic meaning of
$\oplus$.

A program is an expression with no free variables and all function
names defined in XXX.  The intended operational
semantics is given in Figure \ref{fig:redsem}, where
$[\overline{e}/\overline{x}]e'$ is the capture-free substitution of
expressions $\overline{e}$ for variables $\overline{x}$ in $e'$.

Evaluation judgement:
\begin{equation*}
\ENV \ent e \beval v \cost \K{}
\end{equation*}

\begin{figure*}
$\hinfer{E:Var}{\ENV \ent x \beval v \cost \KVAR}{\ENV(x) = v}$ \nl

$\hinfer{E:ConstInt}{\ENV \ent n \beval n \cost \KINT}{}$ \spc
$\hinfer{E:ConstIx}{\ENV \ent i \beval i \cost \KIX}{}$ \spc
$\hinfer{E:ConstB}{\ENV \ent b \beval b \cost \KBOOL}{b \in \{\TRUE, \FALSE \} }$ \nl

$\hinfer{E:ConstA}{\ENV \ent \ARRAY{v_0, \ldots, v_{n-1}} \beval \ARRAY{v_0, \ldots, v_{n-1}} \cost \KARR}{}$ \nl

$\hinfer{E:BinOp}{\ENV \ent x_1 \oplus x_2 \beval v \cost \KOP}{\ENV(x_1) = n_1 \spc \ENV(x_2) = n_2 \spc v = n_1 + n_2}$ \spc
$\hinfer{E:App}{\ENV \ent f(x_1,\ldots,x_n) \beval v \cost \KAPP{1} \KADD \K{} \KADD \KAPP{2}}{\ENV(x_i) = v_i \spc \ENV, x_i = v_i \ent e_f \beval v \cost \K{} \spc i=1\ldots n}$ \nl

$\hinfer{E:Pair}{\ENV \ent \PAIR{x_1}{x_2} \beval \PAIR{v_1}{v_2} \cost \KPAIR}{\ENV(x_1) = v_1 \spc \ENV(x_2) = v_2}$ \nl
$\hinfer{E:Fst}{\ENV \ent \FST x \beval v_1 \cost \KFST}{\ENV(x) = (v_1, v_2)}$ \spc
$\hinfer{E:Snd}{\ENV \ent \SND x \beval v_2 \cost \KSND}{\ENV(x) = (v_1, v_2)}$ \nl

$\hinfer{E:Let}{\ENV \ent \LET x = e_1 \IN e_2 \beval v \cost \KLET{1} \KADD \K{1} \KADD \KLET{2} \KADD \K{2} \KADD  \KLET{3}}{\ENV \ent e_1 \beval v_1 \cost \K{1} \spc \ENV,x_1 = v_1 \ent e_2 \beval v \cost \K{2}}$ \nl

$\hinfer{E:IfT}{\ENV \ent \IF x \THEN e_1 \ELSE e_2 \beval v \cost \KCONT{1} \KADD \K{} \KADD \KCONT{2}}{\ENV(x) = \TRUE \spc \ENV \ent e_1 \beval v \cost \K{}}$ \spc
$\hinfer{E:IfF}{\ENV \ent \IF x \THEN e_1 \ELSE e_2 \beval v \cost \KCONF{1} \KADD  \K{} \KADD  \KCONF{2}}{\ENV(x) = \FALSE \spc \ENV \ent e_2 \beval v \cost \K{}}$ \nl

$\hinfer{E:Par}{\ENV \ent \PARALLEL x\,i.e \beval \ARRAY{v_0\ldots v_{n-1}} \cost \KPAR{1} \KADD \K{j} \KADD \KPAR{2}}{\ENV(x) = n \spc \ENV,i = j \ent e \beval v_j \cost \K{j} \spc j=0\ldots n-1}$ \spc
$\hinfer{E:Index}{\ENV \ent x_1 \INDEX x_2 \beval v_i \cost \KINDEX}{\ENV(x_1) = \ARRAY{v_0,\ldots,v_{n-1}} \spc \ENV(x_2) = i}$ \nl

$\hinfer{E:For}{\ENV \ent \FOR x\,y\, i.s.e \beval v_n \cost \KFOR{1} \KADD \K{j} \KADD \KFOR{2}}{\ENV(x) = n \spc \ENV(y) = v_0 \spc \ENV,i = j, s = v_j \ent e \beval v_{j+1} \cost \K{j+1} \spc j=0\ldots n-1}$
\caption{Reduction semantics}
\label{fig:redsem}
\end{figure*}


FIXME: Rephrase the theorem to fit our semantics which doesn't return the
extended environment.

\begin{theorem}[Immutable values]
Let {\it e} is an expression and $\ENV$ an environment. If $\ENV \ent e \beval v \cost \K{}$ then $\ENV(x) = \ENV'(x)$ for all $x \in \dom(\ENV)$
\end{theorem}


\subsection{Partial semantics}

Convergent evaluations set an upper bound for partial evaluations:
\begin{theorem}[Bounded resource usage]
If $\ENV \ent e \beval v \cost (q, q')$  and $\ENV \ent e \beval \cost p$ 
then $p \leq q$.
\end{theorem}



\section{Type System}

TODO: Explain $\shift$.


% This does not work. Yet.
\begin{figure*}
$\hinfer{T:Par}{\TENV; Q \ent \PARALLEL x\,i.e : (\TARRAY t, Q')}{\TENV(x) = \TINDEX \spc \TENV,i : \TINDEX; Q \ent e : (t, Q'') \spc Q = \shift_{arr}(Q') + \KPAR{0}}$ \nl

$\hinfer{T:For}{\TENV; Q \ent \FOR x\,y\, i.s.e : (t, Q')}{\TENV(x) = \TENV(y) = \TINDEX \spc \TENV,i:\TINDEX, s:t; Q \ent e : (t, Q'')}$
\caption{Typing Rules}
\label{fig:typesyst}
\end{figure*}

\section{Related Work}

\section{Conclusions}

\acks


\bibliographystyle{plainnat}

\bibliography{rawfeldspar}
\end{document}

